\chapter{Literature Review}
\label{chap:litreview}
\lhead{\emph{Literature Review}}
The key question to answer in this chapter is: "What has been done/is being done". 

You must present a summary of background research - what has been done in this area; any related work. Provide reader with information to understand the rest of the report. This chapter comprises around 2000 words and should put your project into context within the field. You must identify the main sources of information you should be aware of within your chosen area and use this activity to strengthen your knowledge in this area. So here you should develop a knowledge of not only your specific topic but also how it is changing. 

Include relevant references. For the (rare) projects with no implementation component, e.g., using neural networks to explore or extend the properties of some mathematical model, or to make predictions of some kind, one might break this chapter into two and write a chapter on neural networks and another on the area of application.

To conduct this activity you should aim to read most of a relevant paper every other day for the remainder of the project, make a brief summary of the main findings and how they relate to your work (less than 30 words). You should continue this activity after your project as CPD.

You should begin your journey of discovery in reverse order to the listing above (which is given in order of academic importance/significance). So when you are researching your topic first look up some TedX talks or youtube tutorials, then research what companies are doing in the area, then get a handful of very good texts on the core topics of your area (anything older than 5 years usually is not helpful here) and finally start reading conference or journal papers (again newer is better here). In particular during this section you may need to use tables to list resources. These are also automatically formatted in latex thus allowing you to concentrate on content. for example table \ref{tab:Mylar}.

\begin{table}[ht]
	\centering
		\begin{tabular}{ c  c  }
		\hline
		\hline
		Parameter & PET \\
		\hline
		Youngs Modulus & 2800-3100MPa \\
		Tensile Strength & 55-75MPa \\
		Glass Temperature & 75$^\circ$C \\
		Density & 1400kg/m$^3$ \\
		Thermal Conductivity & 0.15-0.24Wm$^{-1}$K$^{-1}$ \\
		Linear Expansion Coefficient & $7\times10^-5$ \\
		Relative Dielectric Constant @ 1MHz & 3\\
		Dielectric Breakdown Strength & 17kVmm$^{-1}$\\
		\end{tabular}
	\caption{PET Physical Properties}
	\label{tab:Mylar}
\end{table}

Make sure you cite\cite{Reference1} as many \textbf{relevant} papers \cite{Reference3}, luckily Latex will take care of the ordering of the citations \cite{Reference2} for you.

The aim here is that you find the trends in your topic (3), and more in general in the area in which your topic resides (2) your project falls under and from these trends you develop your initial project question further and begin to get insights into how others have solved/approached similar problems. Think of this section as colouring in your initial idea. Before you approach this section you should read at least 4/5 good review papers. Also remember if you find that you are detailing topics that you have not introduced earlier in this section you need to add a reference for background reading.