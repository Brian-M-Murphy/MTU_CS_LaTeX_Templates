\chapter{Introduction}
\label{chap:intro}
\lhead{\emph{Introduction}}

This chapter should provide background description for your project. This chapter should comprise around 750 words. Here you are setting the scene, remember the reader may know little about your project at this stage (other than the abstract). N.B. The chapter sections outlined in this document are suggested, some projects will have a greater or lesser emphasis on different sections or may change titles and some will have to add other sections to provide context or detail.
% Putting in comments within the TeX file can be really useful in making notes for yourself and dumping text that you intend to edit later

\section{Motivation}
Provide motivation for reading the report. Why is it important to do a project on this topic? This should cover your key motivation for this. This section can include the personal pronoun but \textbf{the rest of the report should be third person passive}, this is the case with most technical reports! For example here it is fine to say "... I decided to develop and app to help ...".

\section{Contribution}
Specify the problem clearly. List your project goals and research questions here. Enumerate the main contributions. How have you advanced the state of the art i.e what have you done that is new?

\section{Structure of This Document}
% notice how I cross referenced the chapters through using the \label tag --> LaTeX is VERY similar to HTML and other mark up languages so you should see nothing new here!
Introduce the structure of your report. This section is quite formulaic. Briefly describe the structure of this document, enumerating what does each chapter and section stands for. For instance in this work in Chapter \ref{chap:litreview} the guidance in structuring the literature review is given. Chapter \ref{chap:design} describes the main requirements for the problem definition and so on ...