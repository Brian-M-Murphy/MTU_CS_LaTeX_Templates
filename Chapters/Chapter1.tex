\chapter{Introduction}
\label{chap:intro}
\lhead{\emph{Introduction}}
This chapter should comprise around 1000 words and introduces your project. Here you are setting the scene, remember the reader may know nothing about your project at this stage (other than the abstract). N.B. The sections outlined in this document are suggested, some projects will have a greater or lesser emphasis on different sections or may change titles and some will have to add other sections to provide context or detail.
% Putting in comments within the TeX file can be really useful in making notes for yourself and dumping text that you intend to edit later

An abbreviation is written as ``\textbackslash gls\{abbreviation\}" \gls{lah} and the first time this is used it will give it in full form. For future uses, it will be given in abbreviated form such as \gls{lah}. It is possible to re-declare again in full form using ``\textbackslash arcfull\{abbreviation\}" \acrfull{lah}. All abbreviations are to be added to the abbreviation file in Information/Abbreviations.tex.

\section{Motivation}
Why is it important to do a project on this topic? This should cover your key motivation for this. For example an excellent student from 2016 noticed a large number of homeless sleeping rough in Cork and was motivated to develop a system that load balanced the homeless shelters to try to accommodate the maximum number of homeless. This section can include the personal pronoun but the rest of the report should be third person passive, this is the case with most technical reports! For example here it is fine to say "... I decided to develop and app to help ...".

\section{Contribution}
Enumerate the main contributions. Here try to zoom out, to talk from the perspective of a Computer Science graduate. In other words, imagine you are talking to a job panel, and you want to show your computer science skills by enumerating how they are reflected in your project work. A good guide here is to look back over the modules you have covered as an undergrad from 2/3rd year, how many tools and techniques from these modules do you have in the project and to what extent? How have you advanced beyond the module content? Do you have anything new?

\section{Structure of This Document}
% notice how I cross referenced the chapters through using the \label tag --> LaTeX is VERY similar to HTML and other mark up languages so you should see nothing new here!
This section is quite formulaic. Briefly describe the structure of this document, enumerating what does each chapter and section stands for. For instance in this work in Chapter \ref{chap:background} the guidance in structuring the literature review is given. Chapter \ref{chap:problem} describes the main requirements for the problem definition and so on ...
