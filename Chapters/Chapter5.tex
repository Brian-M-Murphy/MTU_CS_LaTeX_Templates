\chapter{Testing and Evaluation}
\label{chap:eval}
\lhead{\emph{Project Testing}}

The goal of this chapter is an objective evaluation of the final system. The evaluation must be quantitative and not qualitative. Indicate how the system was verified/tested. You may perform qualitative evaluation but this should not form the basis of the main conclusions you derive from the evaluation. This evaluation, where possible, should be comparative, i.e. you should evaluate your system against a commercially available system and/or system detailed in a research publication. If no implementation was done, you could describe the types of experiments/simulations that were carried out. Explain why these and not others were performed and discuss the most important parameters in the tests/simulations and describe their roles.

You should demonstrate operational testing of the project using real or contrived data sets to evaluate aspects of the project not encompassed in the software testing (e.g. quantify how well does your project achieved the overall goal). 
\begin{itemize}
    \item For software based projects this will include, but should not be limited to, evaluation of non-functional requirements.
    \item For infrastructural projects this testing should include system/network KPI analysis.
    \item For analysis based projects (ML, malware or other) this may include model evaluation or YARA rule validation, for example.
    \item For management projects, where software testing or infrastructure testing may not be in scope, the test process for the system is expected to be more rigorous and well described than a project incorporating significant development work.
\end{itemize} 

Summarize the output data, and the statistical or other techniques to deduce your results. Summarize your results, including tables or graphs as appropriate with a brief description of each. Here, where possible, compare your results with other products/systems. Identify any possible threats to the validity of your results, and discuss each briefly here (you will discuss in more detail in the next chapter).